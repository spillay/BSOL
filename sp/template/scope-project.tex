\section{Project Scope}
The project consists of two key facets; namely a research component and an applied component. The research component would deliver a set of \textit{Model Types} in a cataloged manner which can be applied to e-government services.
The applied component would develop the supporting infrastructure to design, develop and implement e-government services based on the \textit{Model Types} for a specific department following an \gls{MDE} approach. In the interest of delivering and meeting objectives of \client; we propose that the implementation be conducted in parallel with the research effort.

\subsection{Project Scope Definition}
\slist
\spit Research Component
\slist
\spit Type System for the different types of government departments in South Africa. This phase would be initiated as a Phd project
\spit E-government services catalog in line with the UN E-government Survey. This phase would be initiated as a PhD project
\spit A mapping between the catalog and the government department type system
\elist
The scope of the work in the applied phase would be restricted to a chosen department or a predefined set of e-services. Also the notion between e-services and department operations would need to be clearly demarcated. 
\spit Applied Component; The applied component is divided into two parts
\slist
\spit E-services Development Infrastructure
\slist
\spit Provide strategic advice and technical support for an open source toolchain to deliver these services; specifically focusing on the Enterprise Level.
\spit Manage the toolchain to take advantage of the deliverables from phase 1 specifically the \textit{Model Types}.
\spit Jointly formulate supporting models for e-government services
\spit Build and deploy Modeling Translation Engines to implement Enterprise Level Technologies 
\elist
\spit Implementation of the Technology Infrastructure required to support the MDA approach
\slist
\spit Implementation of a GUI generator for front-ends that can cope with Enterprise Requirements; such a server side processing versus client side processing.
\spit Analyze the generic components to ensure their suitability for the framework; the assumption is that these components would be used as is and a separate project would be scoped to optimize these components
\spit Design and Implement a Model Driven Online Help facility
\spit Configure and Implement Integration Service Layer
\spit Design and Implement the ORM Facilities; Automate as far as possible
\spit Configure and Implement Single-sign-on and accounting facilities
\spit Configure and implement Web-services security layer
\elist
\elist
\elist
\subsection{Project Objectives}
Deliver a java platform to support the delivery of e-government services at an enterprise level. This would include the delivery of a enterprise development stack and a technology stack; to provide   model driven approach to deliver on e-government services.  Note that merely building the services and not taking into account the fact that these services would need to serve millions of users at a time, specialized features need to be included in the java platform to ensure that these services can be scaled and well maintained. 
\subsection{Project Deliverables}
The following project deliverables shall be made available to the \client.
\slist
\spit Research Component
\slist
\spit Well defined Catalog for Government Types {As part of a PhD project}
\spit Well defined Catalog E-Government Services {As part of a PhD project}
\spit Mapping based on the South African Context {As part of a PhD project}
\elist
\spit Applied Component
\slist
\spit Enterprise Development Stack
\slist
\spit Define a Domain Specific Language that would support the set of e-government services.
\spit Automate the generation of the UI from the DSL to support multiple targets
\spit Automate the generation of the Business Rules from the DSL with it's Constraint Language exploiting SOA
\spit Automate the generation of a Data Structure together with it's supporting Object Relation Mapping (ORM)
\elist
\spit Technology Stack: Which would be developed in respect to the scope defined in terms of the Project Scope Definition Section. Furthermore this layer would be concerned with fulfilling the needs from a Enterprise Level. Thus these around scalability and redundancy would be key concerns.
\slist
\spit Design/Define and Implement a Presentation Layer 
\spit Design/Define and Implement a Business Process Layer
\spit Design/Define and Implement a Services Layer
\spit Design/Define and Implement a Service Component Layer
\spit Design/Define and Implement a Operational Systems Layer
\elist
\elist
\elist
\subsection{Critical Success Factors}
\vendor shall provide a team of specialists according to the skill contingent as it is stated in paragraph 1.3.  The client \client is requested to make functionaries available on a scheduled basis as and when their input is required.  The following is a definite minimum requirement for the successful completion of the project:
\slist
\spit Adequate work space:  A boardroom or similar room is required to accommodate all officials of the respective \vendor business unit.  
\spit A large white board:  At least one wall mounted, large white board is required for process modelling purposes.  A small stand alone white board will not be suitable.
\spit Business knowledge:  Officials with the necessary business knowledge and knowledge about the requirement must be available.  If this knowledge is not available the \client project manager will immediately be informed and alternative arrangements shall be made.
\spit Work session preparation: The client, \client is requested to prepare themselves and to avail all functional information and documentation to be used during the execution of the project.  The following must please be done:
\slist
\spit Provide a system wide overview of the technical environment
\spit Provide all relevant documentation in terms of Technology Stack
\spit Provide information to any master data that currently exists
\spit Provide information in terms of requirements gathering processes

%\spit Confirm the already identified objectives, business and system objectives.
%\spit Confirm the scope as stated in the proposal.
%\spit Determine the official business processes that are currently being performed within the general invoice business environment seeing that it will form the focus point of the URS exercise.    
%\spit Obtain all functional business documents that are in use during the execution of the above business process.  These documents must be available during the URS work sessions.
%\spit A summary of pre-determined requirements is to be compiled and be available for discussion during the URS work sessions. 
%\spit All identified changes and deviations from the proposal must be documented and also be available at the commencement of the URS exercise.
\elist
\spit Work session scheduling:  The scheduling of work sessions with client representatives is an important activity.  \vendor expects a four hour turnaround time for clients to respond to work session invitations.   
%\spit Work session duration: Depending on the [client] officials, it is foreseen that no more than three work sessions will be needed to perform business process modelling to determine business and user requirements for the Track and Trace System.  Work sessions will be conducted from 09:00 until ±12:00.
\spit Client availability: \vendor determines fixed target dates for the completion of all deliverables of a project per formal, client accepted, project plan.  If target dates cannot be met due to client un-availability, the client will carry the \vendor resource cost for the time lost as a financial penalty.       
\spit Parking arrangement: Secure vehicle parking must be made available to the \vendors personnel.
%\spit Alternate office space: The SITA offices at Perseus Park, in Lynwood, is available for the execution of the project should it be required by the client.
\elist
