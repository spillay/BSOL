\clearpage
\section{Project Structure}
\subsection{Stakeholders}
\begin{enumerate}
	\item \client
	\item \vendor
\end{enumerate}
\subsection{Project Steering Committee}
\begin{enumerate}
	\item \client Project Manager
	\item \vendor Project Manager or Project Director
	\item \client Technical Manager
	\item \vendor Technical Manager
\end{enumerate}
Additional members shall be identified, if needed, and incorporated into the steering committee. The steering committee shall be the primary decision making forum for the project and they shall provide direction and guidance to the project. Key members of the steering committee must have full delegation of authority for decision making.
\\
All major project deliverables shall be presented to the user representatives for quality review and acceptance, before submission to the steering committee for approval. 
\subsection{Project Management - Teams and Responsibilities}
The project management team shall consist of the \vendor project manager and the \client project manager and their responsibilities are as follows:
\begin{enumerate} 
\item Executing tasks in terms of the project charter and plan and in accordance with the scope, and supporting the service's objectives as outlined in this document.
\item Ensuring that all project deliverables are met on time and within the budget/quotation.
\item Identifying and reporting on project dependencies.
\item Identifying and minimizing the effects of identified project constraints.
\item Co-coordinating the responsibilities of \client in support of this proposal.
\item Managing identified and new project	 risks continuously.
\item Reporting on progress and project risks to the project steering committee.
\end{enumerate}
\subsection{\vendor  project team responsibilities}
\begin{center}
    \begin{tabular}{ | l | p{10cm} |}
    \hline
    Resource Type & Responsibilities \\ \hline
Project manager &
Manage all aspects of project delivery, including resource management and liaising with the client regarding project schedules. Report on progress and project risks. Distribute all management reports. Schedule all review sessions and co-ordinate internal meetings. Internal \vendor liaison and project financial management. \\ \hline
    \hline
Project co-ordinates &
Shall be responsible for the following:
\slist
\spit project administration tasks such as taking minutes;
\spit scheduling \client related meetings;
\spit distributing quality records, deliverables; and
\spit liaising with researchers and engineers.
\elist \\ \hline
    \end{tabular}
\end{center}

\subsection{\client  project team responsibilities}
\begin{center}
    \begin{tabular}{ | l | p{8cm} |}
    \hline
    Resource Type & Responsibilities \\ \hline
\client business unit officials/functionaries &
Will be responsible to:
\slist
\spit have all related information available regarding the user and business requirements.
\spit avail themselves according to the scheduled work session
\elist \\ \hline
Project manager &
Will be responsible to:
\slist
\spit schedule internal \client review sessions.
\spit Facilitate the document approval process at the \client.
\spit coordinate meetings with \client members. 
\spit monitor \client time schedules of deliverables.
\spit attend project meetings.
\spit ensure the completion of minutes of meetings.
\spit identify risks and escalate them to the steering committee if necessary.
\elist \\ \hline
    \hline
    \end{tabular}
\end{center}

\subsection{Controls Strategy}
\subsubsection{Meetings}
\slist
\spit Project progress meetings shall take place at least bi-weekly between \client and \client in order to discuss progress reports, and the quality and progress of services rendered in terms of the project charter and plan. Actions that need to be addressed during the weeks to follow shall also be discussed, as well as the invoices. Minutes of all meetings shall be kept in the project journal that is kept with the \client project manager. \client shall chair these meetings.
\spit Project team meetings shall take place weekly in order to discuss execution issues. 
\spit Quality review meetings shall take place to review all documented deliverables internally, prior to submission to \client for formal review and approval. Quality records shall be kept in the project journal.
\spit The steering committee shall meet monthly or when required, to resolve specific issues that could have an impact on the project. Relevant members shall be invited to attend these meetings as and when required. \client shall chair these meetings. \client shall compile the minutes and distribute them within five working days after the meeting has taken place.
\elist
\subsubsection{Reports}
A monthly project progress report shall be presented to \client on all tasks executed in terms of this proposal. All approved services that might influence the total costing of this proposal, shall be outlined in the report. Risks and issues shall also be addressed as part of the report. The exact content and detail of the report shall be tailored to suit the operational requirements of the \client; which must be agreed upon by the \vendor and \client at the initiation stage of the project.
\subsubsection{Quality Assurance}
\vendor shall ensure that all deliverables are in accordance with the \client standards, to be agreed upon on initiation and
Quality records shall be kept by \vendor and \client.
\subsubsection{Document Control}
\slist 
\spit All documentation shall be classified as "confidential" and be available from the \client project manager.
\spit The \client configuration office shall maintain formal records of changes to any approved documentation.
\spit All project-related work papers shall be available from \clients project manager.
\spit Project documentation shall be published on the relevant \client portals by the \client project manager.
\elist
\subsubsection{Configuration Management}
\slist
\spit \client shall ensure proper configuration management on all products, services and service deliverables that relate to this proposal and subsequent charter. Official documents/project deliverables shall be controlled according to \clients corporate standards and procedures.
\spit \clients project manager is accountable for distributing controlled copies. 
\elist
\subsubsection{Risk Management}
\slist
\spit A formal risk management approach shall be adopted in accordance with the \client standard for project risk management. \client is expected to give guidance in terms of this register.
\spit High-probability and high-impact risks shall be monitored throughout the project, and appropriate risk management actions implemented to reduce the probability and/or impact of these risks. Risks shall be documented in the monthly progress report.
\spit A formal project risk and issue meeting maybe held separately in order to ensure a focused approach to risk mitigation.
\elist
\subsubsection{Risk Assessment}
\begin{center}
    \begin{tabular}{ | l | l | p{5cm} |}
    \hline
    No & Risk Description & Risk Containment Measure \\ \hline
1. & Project scope creep & If any change to the original URS causes a deviation from the anticipated effort to complete the task, re-planning of the time and expense budget shall be compiled and submitted for user approval. \\ \hline
2. & Approving and signing-off  deliverables/phases & Delay in approving or signing-off a deliverable by the \client might cause a delay in the continuation of other deliverables, resulting in a delay in completing the project. Exceptions shall be escalated to \client for a resolution. Bi-weekly monitoring of the project plan, in conjunction with \client, shall minimize the occurrence of this risk. \\ \hline
3. & Availing funds for the duration of the project & In the event a delay is caused outside the control of \client, the funds necessary to cover the extra expense shall be negotiated with \client. \\ \hline
4. & Changing the delivery date & If \client requires project deliverables to be completed before the planned delivery date, additional resources shall be allocated to the project and/or the allocated resources shall work overtime. This could cause a baseline change that could result in additional costs. It is the client's \client responsibility to oversee that the planned project dates are not shortened. \\ \hline

    \hline
    \end{tabular}
\end{center}


